\documentclass[]{article}   %样式模板
\usepackage{ctex}  %切换为中文
%opening
\title{Hello TexStudio}  %文章标题
\author{Peitsan}    %文章作者
\begin{document}  %Tex体
	\maketitle
	\begin{abstract}  %摘要
		This is abstract.This is abstract.This is abstract.This is abstract.This is abstract.This is abstract.This is abstract.This is abstract.This is abstract.This is abstract.This is abstract.This is abstract.This is abstract.This is abstract.This is abstract.This is abstract.This is abstract.
		
		
		这是摘要这是摘要这是摘要这是摘要这是摘要这是摘要这是摘要这是摘要这是摘要这是摘要这是摘要这是摘要这是摘要这是摘要这是摘要这是摘要这是摘要	这是摘要这是摘要这是摘要这是摘要这是摘要这是摘要这是摘要这是摘要这是摘要这是摘要这是摘要这是摘要这是摘要这是摘要这是摘要这是摘要这是摘要	这是摘要这是摘要这是摘要这是摘要这是摘要这是摘要这是摘要这是摘要这是摘要这是摘要这是摘要这是摘要这是摘要这是摘要这是摘要这是摘要这是摘要	这是摘要这是摘要这是摘要这是摘要这是摘要这是摘要这是摘要这是摘要这是摘要这是摘要这是摘要这是摘要这是摘要这是摘要这是摘要这是摘要这是摘要	这是摘要这是摘要这是摘要这是摘要这是摘要这是摘要这是摘要这是摘要这是摘要这是摘要这是摘要这是摘要这是摘要这是摘要这是摘要这是摘要这是摘要
	\end{abstract}
	
	\newpage
	
	\tableofcontents %这是目录
	
	\newpage
	\section{问题重述}
	\subsection{问题的背景}
	\subsection{问题的提出}
	
	\noindent\textbf{1.2.1 问题一}
	
	\noindent\textbf{1.2.2 问题二}
	
	\noindent\textbf{1.2.3 问题三}
	
	\section{问题分析}
	\subsection{问题一的分析}
	\subsection{问题二的分析}
	\subsection{问题三的分析}
	\section{模型假设}
	
	\section{符号说明}
	%	表格语法下节课再讲
	\begin{center}
		\begin{tabular}{cc}
			\hline
			\makebox[0.45\textwidth][c]{符号}	& \makebox[0.45\textwidth][c]{意义}   \\ \hline
			Symbol  & Meanings \\ \hline
		\end{tabular}
	\end{center}
	
	\begin{center}
		\begin{tabular}{ccc}
			\hline
			\makebox[0.25\textwidth][c]{符号}	& \makebox[0.35\textwidth][c]{意义} & \makebox[0.2\textwidth][c]{单位}	 \\ \hline
			Symbol  & Meanings & Units\\ \hline
		\end{tabular}
	\end{center}
	\newpage
	\section{模型的建立与求解}
	\subsection{问题一的模型}
	
	\noindent\textbf{5.1.1 数据预处理}
	
	\noindent\textbf{5.1.2 模型的建立}
	
	\noindent\textbf{5.1.3 模型计算求解}
	
	\noindent\textbf{5.1.4 问题一的结论}
	\subsection{问题二的模型}
	\subsection{问题三的模型}
	\newpage
	\section{模型的验证}
	\subsection{问题一的检验}
	\subsection{问题一的检验}
	\subsection{问题一的检验}
	\newpage
	\section{模型的评价与改进}
	\subsection{模型的优点}
	\subsection{模型的不足}
	\subsection{模型的改进}
	
	\appendix 
	\renewcommand{\appendixname}{Appendix~\Alph{section}}
	\section{附表1——XXXXXX}
	
	\newpage	
	\begin{thebibliography}{9}%宽度9
		\bibitem{bib:one}
		\bibitem{bib:two}
	\end{thebibliography}
	
\end{document}