% !Mode:: "TeX:UTF-8"
% !TEX program  = xelatex
\documentclass{cumcmthesis}
\usepackage{gbt7714}
\usepackage{array}
\usepackage{booktabs}
\bibliographystyle{gbt7714-numerical}
%\documentclass[withoutpreface,bwprint]{cumcmthesis} %去掉封面与编号页
\title{}
\tihao{A}            % 题号
\baominghao{4321}    % 报名号
\schoolname{重庆邮电大学}
\membera{}
\memberb{}
\memberc{}
\supervisor{沈世云}
\yearinput{2022}     % 年
\monthinput{09}      % 月
\dayinput{13}        % 日

\begin{document}
	\maketitle
	\begin{abstract}
		。摘要的具体内容。摘要的具体内容。摘要的具体内容。摘要的具体内容。
		\keywords{关键词1\quad  关键词2\quad   关键词3}
	\end{abstract}
	%        \tableofcontents
	\section{问题重述}
	\subsection{问题背景}

	\subsection{问题提出}
	\noindent\textbf{问题一:}

	\noindent\textbf{问题二:}
	
	\noindent\textbf{问题三:}

	
	\noindent\textbf{问题四:}

	
	\section{模型的假设}
	
	1.假设矿石加工各环节没有先后之分,只考虑温度影响,工艺中电压、水压等其他参数均不变。
	
	2.假设调温指令间隔不短于2小时,且施令后的瞬间可以认为温度已达到设定温度;系统温度与调温指令设定值稳定相近,允许有轻微波动。
	
	3.假设对于不同类型的数据,采集时间间隔不同,数据采集互不干扰。
	
	4.假设原矿参数和产品目标质量已知,系统温度未知;过程数据真实反映原矿质量。
	
	5.假设单位时间生产的产品数量相同,合格率为合格产品数与产品总数的比值。
	
	\section{符号说明}
	
	\begin{center}
		\begin{tabular}{ccc}
			\hline
			\makebox[0.25\textwidth][c]{符号}	& \makebox[0.35\textwidth][c]{意义} & \makebox[0.3\textwidth][c]{单位}	 \\ \hline
			D	    & 木条宽度 &(cm) \\ \hline
		\end{tabular}
		
	\end{center}
	\section{问题分析}
	
	\subsection{考虑系统温度相近产品质量也可能有比较大的差别}
	
	
	\subsection{考虑同一组产品质量可能有多种调温方法都可以得到}
	\subsection{考虑预测产品的合格率}
	
	\subsection{考虑产品的}
	
	\section{数据预处理}
	\subsection{附件1的数据清洗}
	\subsection{附件2的数据清洗}
	
	\section{问题一模型}
	
	\subsection{问题一的数据清洗}
	
	\subsection{图像温度数值识别模型建模过程}
	
	\subsection{模型计算求解}
	
	\subsection{问题一模型的结论}
	\section{问题二模型}
		\subsection{问题二的数据清洗}
	\subsection{模型计算求解}	
	\section{问题三模型}
		\subsection{问题三的数据清洗}
	\subsection{模型计算求解}
	\section{问题四模型}
	
	\s\ar[ul]^-{u}bsection{模型计算求解}
	
	\section{模型的评价和改进}
	
	\subsection{模型的优点和不足}	
	
	\noindent\textbf{10.1.1.问题一的模型评价}
	
	\noindent\textbf{10.1.2.问题二的模型评价}
	
	\noindent\textbf{10.1.3.问题三的模型评价}
	
	\noindent\textbf{10.1.4.问题四的模型评价}
	
	\subsection{模型的改进}	
	
	\noindent\textbf{10.2.1.问题一的模型改进}
	
	\noindent\textbf{10.2.2.问题二的模型改进}
	
	\noindent\textbf{10.2.3.问题三的模型改进}
	
	\noindent\textbf{10.2.4.问题四的模型改进}
	
	\cite{s}
	\bibliography{bib/cite.bib}
	%        \begin{thebibliography}{9}%宽度9
	%            \bibitem{bib:one} ....
	%        \end{thebibliography}
	\newpage
	\begin{appendices}
		\section{文件列表}
		% Table generated by Excel2LaTeX from sheet 'Sheet1'
		\begin{table*}[htbp]
			\renewcommand\arraystretch{0.1}
			\centering
			\caption{支撑材料列表}
			\begin{tabular}{ccc}
				\toprule[1.5pt]
				\makebox[0.27\textwidth][c]{文件夹名}	& \makebox[0.3\textwidth][c]{文件夹内容}	& \makebox[0.4\textwidth][c]{文件描述} \\ 
				\midrule
			&	Data1.mat & 附件1数据 \\
			&	Data2.mat & 附件2数据 \\
			&	Data3.mat & 附件3数据 \\
			&	problem1.m & 问题1求解h \\
				\bottomrule
			\end{tabular}%
			\label{tab:addlabel}%
		\end{table*}%

		
		\section{附表:问题结果表格}
		% Please add the following required packages to your document preamble:
		% \usepackage{booktabs}

		\section{问题一--matlab 源程序}
		
		\lstinputlisting[language=matlab]{code/test3.m}
		\section{问题二--python 源程序}
		test3.m
		\lstinputlisting[language=matlab]{code/test3.m}
		
		\section{问题三--lingo源代码}
		
		\lstinputlisting[language=matlab]{code/test3.m}
	\end{appendices}
\end{document}
